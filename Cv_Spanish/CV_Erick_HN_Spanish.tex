%! Author = erick-hdz
%! Date = 21/03/20

% Preamble
\documentclass[twoside,a4paper,openright,10pt]{report}
% Packages
\usepackage[spanish]{babel}
\usepackage[latin1]{inputenc}
\usepackage{colortbl}
\usepackage{enumitem}
%Setter Methods
\setlength{\textwidth}{190mm}
\setlength{\textheight}{270mm}
\setlength{\oddsidemargin}{-15mm}
\setlength{\evensidemargin}{-15mm}
\setlength{\topmargin}{-30mm}
% Style the page
\pagestyle{empty}


% Document
\begin{document}
\begin{table}[ht]
\centering
    \begin{tabular}{p{40mm} p{140mm}}
        \multicolumn{2}{l}{\textbf{Erick Hernandez Navarrete}, Nacionalidad Mexicana(ID Oficial MX)}\\
        \multicolumn{2}{l}{20 Cto Plaza de la paja N°20,CDMX, México}\\
        \multicolumn{2}{l}{5522357333}\\
        \multicolumn{2}{l}{\texttt{erikh93@ciencias.unam.mx}}\\ \\
        \multicolumn{2}{c}{\textbf{\textit{\large Matematicas y ciencias de la Computacion:
        Teoricamente me gustan los modelos De computo Formales asi como la implementacion en Ingenieria De Software}}}\\
        \multicolumn{2}{c}{\textbf{\textit{\large Mi Objetivo es Envolverme en el flujo de Trabajo y su desarrollo como Codigo}}}\\

        \multicolumn{2}{c}{\cellcolor{black} \textcolor{white}{Educación}}\\ \\

        \textbf{2007-2010}& Escuela Tecnica en Programación
        \textbf{CECY'T 4 'Lazaro Cardenas del Rio' IPN Mexico.}\\
        \textbf{2011-2015} & Licenciatura en Matematicas en \textbf{ Facultad de ciencias,UNAM, Mexico.}\\
        \textbf{2016-Present}& Maestria en ciencias de la Computacion
        \textbf{ IMATE, IIMAS UNAM}.\\
        \\
        \multicolumn{2}{c}{\celcolor{black}\textcolor{white}{Experiencia Laboral}}
        \textbf{2015-2019}& Ayudante de Asignatura en Diversas Asignaturas en la Facultad
        de ciencias de la UNAM\textbf{Facultad de ciencias de la UNAM,UNAM}, \textbf{México}\\
        \textbf{2018-Present} & Desarrollador Backend y analista Cuantivativo en la empresa \textbf{Inteligencia Capital, Finbitz}
        \textbf{México}. Desarrollo de Infraestructura como codigo en los Siguientes Puntos:\\
        & \vspace{-2mm} \begin{itemize}[noitemsep,noliststep]
                            \item Capas De Software para la simulacion e Implementacion de Optimizacion de Portafolios de Inversion
                            \item Desarrollar Rest_APIS y APIS internos de Desarrollo tal como finbitz_API
                            \item Desarrollo de la Base de Datos asi como su respectivo Mantenimiento
                            \vspace{-4mm}
        \end{itemize}\\

        \\
        \multicolumn{20}{c}{\cellcolor{black} \texcolor{white}{Habilidades}}\\ \\
        %Skills in the all the context
        \textbf{Diseño de Algoritmos (Modelo RAM(Maquinas de Turing)), &\textbf{Temporalmente}, &\textbf{Espacialmete}}\\
        \textbf{Lenguajes de Programacion}& Manejo de Lenguajese programación en bajo Nivel y un alto Nivel:C,C++,Java,Python, para hacer Abstracciones De Software;
        Manejo de Lenguajes de capa declarativa, como son SQL para Bases de Datos Relacionales.\\
        \textbf{Sistemas Operativos}&:Diferentes Ramas de Linux Como son: Ubuntu, Kubuntu,Debian y KaliLinux\\
        \textbf{Manejo de CSV}&:Manejo de Invocaciones De Subrutinas de la Capa de Git\\
        \textbf{Manejo de Redes De Computadoras}& Manejo de las Capas De Redes De Computadoras de todas las Capas de Abstraccion:
        tales como son Transporte, Protocolos De Comunicacion HTTPS,HTTP y manejos de Trafico en las capas de Redes como son:TCPDUMP y WIRESHARK\\
        \textbf{Frameworks y lenguajes De programacion, tales como son Markups}&\vspace{-2mm}
        &\begin{itemize}[noitemstep][noliststep]
             \item XML
             \item HTML
             \item JavaScript
             \item jinja Templates
        \end{itemize}\\
        \textbf{Lenguages}&\textbf{Español:}Lengua Natural.\\
        &\textbf{Ingles:}Lengua aprendida:Nivel Intermedio-Avanzado, Habla:FLuida\\
        &\textbf{Frances:}&Nivel Muy Basico\\
        &Trabjo en Un Ambiente Internacional\\

        \\
        \multicolumn{2}{c}{\cellcolor{black} \textcolor{white}{Intereses y Actividades}}\\ \\
        \textbf{Tecnologia Computacional}\\
        \textbf{Artes Marciales}\\
        \textbf{Programacion}\\
        \textbf{Computacion Teorica}\\

    \end{tabular}
\end{table}
\end{document}