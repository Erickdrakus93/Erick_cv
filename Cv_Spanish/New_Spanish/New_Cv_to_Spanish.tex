\documentclass[twoside,a4paper,openright,10pt]{report}
\usepackage[spanish]{babel} % This is the lang in the next context
\usepackage[latin1]{inputenc}
\usepackage{colortbl}
\usepackage{enumitem}
\usepackage{graphicx}

\setlength{\textwidth}{190mm}
\setlength{\textheight}{270mm}
\setlength{\oddsidemargin}{-15mm}
\setlength{\evensidemargin}{-15mm}
\setlength{\topmargin}{-30mm}

\pagestyle{empty}

\begin{document}

\begin{table}[ht]
\centering
\begin{tabular}{p{40mm} p{140mm}}

\multicolumn{2}{l}{\textbf{Erick Hernandez Navarrete},Nacionalidad Mexicana(ID oficial)}\\
\multicolumn{2}{l}{20 Cto.Plaza de la Paja, 09020,CDMX, Mexico}\\
\multicolumn{2}{l}{55 50 67 31 74}\\
\multicolumn{2}{l}{\texttt{erikh93@ciencias.unam.mx}}\\ \\
\multicolumn{2}{c}{\textbf{\textit{\large Matematicas y ciencias de la Computacion y su implementacion en Sistemas Reales.}}}\\
\multicolumn{2}{c}{\textbf{\textit{\large Mi Objetivo es envolverme de los asuntos del Trabajo hasta hacer el Codigo.}}}\\ \\

\multicolumn{2}{c}{\cellcolor{black} \textcolor{white}{Educacion}}\\ \\

 \textbf{2007-2010} & Escuela Secundaria con Especialicacion en Programacion
\textbf{CECY'T 4 'Lazaro Cardenas del Rio' IPN Mexico.}\\
 \textbf{2011-2015} & Licenciatura en Matematicas. \textbf{ Facultad de Ciencias,UNAM, Mexico.}\\
 \textbf{2017-Present} & Maestria en Ciencias de la Computacion
 \textbf{ IMATE, IIMAS UNAM} .\\
\\

\multicolumn{2}{c}{\cellcolor{black} \textcolor{white}{Experiencia Laboral}}\\ \\

\textbf{2015-2017} & Ayudante de Asignatura en Diversas Materias de la carrera de Matematicas en  \textbf{Facultad de ciencias,UNAM}, \textbf{Mexico}.\\

\textbf{2018-Present} & Desarrollador Backend y analista de Datos Cuantitativo en  \textbf{Inteligencia Capital y Finbitz}, \textbf{Mexico}. Desarrollo de Codig como infraestructura en:\\
& \vspace{-2mm} \begin{itemize}[noitemsep,nolistsep]
\item Capas de Software para la simulacion e Implementacion de Optimizacion de Portafolios de Inversion
\item Desarrollo de RestAPIS y APIS internos tales como FBZAPI
\item Desarrollo y creacion de Bases de Datos asi como su respectivo mantenimiento.
\vspace{-4mm}
\end{itemize}\\

\\
\multicolumn{2}{c}{\cellcolor{black} \textcolor{white}{Experiencia en Ivestigacion Universitaria}}\\ \\

\textbf{2015- 2016} & Investigacion Academica en Teoria de Nudos, Matematicas Puras, UNAM.\\
\textbf{2017- 2018} & Proyecto Academico en desarrollo de Algoritmos en los Mercados Financieros con Python Dr.Ricardo Mancilla,UNAM.\\
\textbf{2019-present} & Proyecto de Maestria en Computo Teorico, "Haciendo un mapeo entre el Mundo Distribuido y el Mundo de Turing", UNAM.\\

\\
\multicolumn{2}{c}{\cellcolor{black} \textcolor{white}{Haibilidades}}\\ \\

\textbf{Lengujes de Programacion} & Manejo de Lenguajes de Programacion de bajo y alto nivel:C,C++,Python,java,Lis,etc para escribir instrucciones o subrutinas en capas de Software o Algoritmos.\\
\textbf{Lengujes de Bases de Datos SQL} & PostgreSQL,MySQL,etc.\\
\textbf{Sistemas Operativos} & Diversas ramas de Linux como son Debian, Ubuntu, Kubuntu, Kali, etc. \\
\textbf{Redes de Computadoras} & Manejo de capas de Redes De Computadoras:Transporte,Comunicacion,Protocolos Https Ssl. etc asi como tambien los Frames de Trafico de Computadoras:Tcpdump,Wireshark.\\
\textbf{Web Development} & Frameworks Web y desarrollo web como es Document Object Model dom:\\
& \vspace{-2mm}
\begin{itemize}[noitemsep,nolistsep]
\item HTML
\item Javascript
\item PHP
\item Jinja Templates
\item Spring
\item Maven
\vspace{-4mm}
\end{itemize}\\
\\

\textbf{Stadistica:}&
\begin{itemize}[noitemsep,nolistsep]
\item Inferencia Stadistica.
\item Stadistica Descriptiva.
\item Series de Tiempo y correlaciones .
\end{itemize}\\


\textbf{Lenguajes} & \textbf{Español:} lengua natural.\\
& \textbf{Ingles:} Lengua aprendida: nivel Intermedio-Avanzado, habla fluida \\
& \textbf{Frances:} Basico.\\
& Tener experiencia en un ambiente Internacional.

\\
\multicolumn{2}{c}{\cellcolor{black} \textcolor{white}{Intereses y Actividades}}\\ \\

\textbf{Tecnologia Computacional}\\
\textbf{Tecnologia}\\
\textbf{Programación}\\
\textbf{Matematicas}

\end{tabular}
\end{table}

\end{document}